\documentclass[a4paper,12pt,oneside]{amsbook}
\usepackage[T1]{fontenc}
\usepackage{textcomp}
%\usepackage{euler}
%\usepackage{garamond}

\usepackage{newcent}
%\fontfamily{fxm}\selectfont
\usepackage{fancyhdr}
\pagestyle{fancy}
% Detta paket skoter grafiken
\usepackage[dvips]{graphicx}
%\usepackage[pdftex]{graphicx}
\usepackage{tabularx}
\usepackage{amsthm}
\usepackage{amsmath}
\usepackage[swedish]{babel}
% For bortkommentering
\usepackage{verbatim}
% For att slippa indentering av Sections och SubSections
\usepackage{indentfirst}

% Har stalls paragrafindentering och avstand mellan stycken och rader mm
\setlength{\parindent}{0pt}
\setlength{\parskip}{6pt}
\setlength{\baselineskip}{12pt}
% Har definieras sidan
\setlength{\voffset}{-15mm}
\setlength{\hoffset}{-5.5mm}

\setlength{\topmargin}{0mm}
\setlength{\headheight}{10mm}
\setlength{\headsep}{8mm}
\setlength{\footskip}{16mm}

\setlength{\evensidemargin}{0mm}
\setlength{\oddsidemargin}{0mm}

\setlength{\marginparsep}{0mm}
\setlength{\marginparwidth}{0mm}

\setlength{\textwidth}{170mm}
\setlength{\textheight}{225mm}
\setlength{\paperwidth}{210mm}
\setlength{\paperheight}{297mm}
\setlength{\headwidth}{170mm}
% Huvud och Fot
%\fancypagestyle{}{
\fancyhf{}\fancyhead[c]{\small{\textit{Svar och r�ttningsanvisningar
      till Programmeringsolympiadens skolkval 2018}}}
\fancyfoot[c]{\thepage}
\renewcommand{\headrulewidth}{0.4pt}
\renewcommand{\footrulewidth}{0.4pt}
\newenvironment{lista}
{\begin{itemize}
\setlength{\parindent}{0pt}
\setlength{\itemsep}{6pt}}
{\end{itemize}}
% Uppgifter
\newcounter{probnr}
\newenvironment{tal}{%
\begin{list}
%{\textbf{\arabic{section}.\arabic{probnr}}} {\usecounter{probnr}
{\textbf{\arabic{probnr}}} {\usecounter{probnr}
\setlength{\leftmargin}{0mm}
\setlength{\rightmargin}{0mm}
\setlength{\labelwidth}{-1mm}
\setlength{\labelsep}{1mm}}
\setlength{\itemsep}{6pt}
}{\end{list}}
% Definition av avsnitt: FAKTA, EXEMPEL, PASTAENDE
\newtheorem{fakta}{Fakta}
\newtheorem{exempel}{Exempel}
\newtheorem{problem}{Problem}
%
\newtheoremstyle{test}% NAME
{20pt}      % ABOVESPACE
{10pt}      % BELOWSPACE
{\sffamily} % BODYFONT
{0pt}       % INDENT
{\scshape}  % HEADFONT
{}          % HEADPUNCT
{\newline}  % HEADSPACE
{}          % CUSTOM-HEAD-SPEC

\theoremstyle{test}
\newtheorem{program}{Program}
\newcommand{\sv}[1]{\textsc{#1}}            % Sma versaler
\newcommand{\fe}[1]{\textbf{#1}}            % FET
\newcommand{\ku}[1]{\textit{#1}}            % KURSIV
\newcommand{\cu}[1]{\texttt{#1}}            % Courier
\newcommand{\sk}[1]{\texttt{#1}}            % Courier
\newcommand{\rubrik}[1]{\begin{center}\sf\huge{#1}\normalsize\rm\end{center}}
\begin{document}
%\DeclareGraphicsExtensions{.jpg,.pdf,.mps,.png,.eps}


\specialsection*{Svar och r�ttningsanvisningar}
\thispagestyle{fancy}
\lhead{}
\begin{itemize}
%\setlength\itemsep{0.2cm}
\item L�s igenom \textit{t�vlingsreglerna}.
\item Programmen tas i tur och ordning in i editorn och kompileras.
Uppst�r kompileringsfel betraktas programmet som felaktigt och l�sningen
ges $0$ po�ng.
\item Programmet k�rs med givna indata enligt nedan. Alternativt, om eleven f�rberett programmet f�r det, kan de bifogade indatafilerna anv�ndas ist�llet.
\item Varje testfall
  med korrekt svar ger 1 po�ng.
\item
Totalt kan man
  allts� f� 5 po�ng f�r varje uppgift.
\item Om exekveringstiden f�r ett testexempel, k�rt p� en vanlig dator,
\textit{�verskrider 3 sekunder} betraktas k�rningen av testexemplet som felaktigt.
\item Det kan vara viktigt att programmet k�rs i en milj� liknande den som
programmet utvecklats i, samma version av kompilator eller
interpretator.
\item Vid problem i samband med r�ttningen �r det viktigt att det sunda
f�rnuftet f�r r�da!
\item Ett f�rslag till r�ttningsprocedur kan vara att l�ta eleven
sitta vid datorn.
\end{itemize}

\newpage


\subsection*{Uppgift 1 -- Tunnelbanan}
~\\
~\\
{\tt 
\begin{tabular}{||l||c|c|c|c||c||}\hline\hline
& \multicolumn{4}{c||}{\fe{Indata}} & \fe{Utdata} \\ 
& $a_1$ & $a_2$ &$a_3$ & $a_4$& \\ \hline \hline
\fe{Test 1} & 0  &5  &10 &20 &33\\ \hline
\fe{Test 2} & 21 &0  &10 &5  &18\\ \hline
\fe{Test 3} & 8  &7  &0  &6  &12\\ \hline
\fe{Test 4} & 69 &81 &15 &2  &71\\ \hline
\fe{Test 5} & 13 &20 &15 &0  &25\\ \hline\hline
\end{tabular}
}
%$$

\subsection*{Uppgift 2 -- K�pa matta}
~\\
~\\
{\tt 
\begin{tabular}{||l||c|c||c|c||}\hline\hline
& \multicolumn{2}{c||}{\fe{Indata}} & \multicolumn{2}{c||}{\fe{Utdata}} \\ 
& $M$ & $N$ & $B$ & $L$ \\ \hline \hline
\fe{Test 1}&700 &750 &27 &27\\ \hline                           
\fe{Test 2}&901 &959 &30 &31\\ \hline                            
\fe{Test 3}&123456789012 &123456789012 &12 &10288065751\\ \hline 
\fe{Test 4}&987654021 &987654321 &31382 &31472\\ \hline          
\fe{Test 5}&987987123123 &987987987987 &993975 &993976\\ \hline 
\hline
\end{tabular}
}

\subsection*{Uppgift 3 -- Flyttkartonger}
~\\
~\\
{\tt 
\begin{tabular}{||l||c|l||c||}\hline\hline
& \multicolumn{2}{c||}{\fe{Indata}} & \fe{Utdata} \\ 
& $N$ & $a_1, a_2, ..., a_N$ &  \\ \hline \hline
\fe{Test 1}&3 & 13 6 20 & 21 \\ \hline                                                                                              
\fe{Test 2}&5 & 77 38 90 19 78 & 253 \\ \hline                                                                                      
\fe{Test 3}&10 & 535 50 273 640 906 152 8 816 292 943 & 138181  \\ \hline                                                           
\fe{Test 4}&20 & 1542 582 2038 140 206 1513 1976 1220 984 1556 &  \\
           &&1159 1461 2626 1437 883 2169 1172 2980 799 103 & 100071064  \\ \hline
\fe{Test 5}&20 & 1264 1097 1057 1914 690 2889 1796 384 1936 747 & \\
           && 2948 939 1673 1473 1775 662 1997 1945 1 3000 & 555493117 \\ \hline  
\hline
\end{tabular}
}

\subsection*{Uppgift 4 -- Brickor}
~\\
~\\
{\tt 
\begin{tabular}{||l||l||c||}\hline\hline
& \fe{Indata} & \fe{Utdata} \\ \hline\hline
\fe{Test 1}&VSVVVS & 2\\ \hline         
\fe{Test 2}&VSSVSSVSSV & 3\\ \hline     
\fe{Test 3}&VSVSVVSVVS & 5\\ \hline     
\fe{Test 4}&VVVSVSVSVSVSVSV & 9\\ \hline
\fe{Test 5}&SSVSVVVSSSVSVSV & 8\\ \hline
\hline
\end{tabular}
}

\subsection*{Uppgift 5 -- DuTub}
~\\
~\\
{\tt 
\begin{tabular}{||l||c|c||c|c||c|c||c|c||c|c||}\hline\hline
& \multicolumn{2}{c||}{\fe{Test 1}} & \multicolumn{2}{c||}{\fe{Test 2}}& \multicolumn{2}{c||}{\fe{Test 3}} & \multicolumn{2}{c||}{\fe{Test 4}} & \multicolumn{2}{c||}{\fe{Test 5}} \\\hline\hline
Antal videor& \multicolumn{2}{c||}{ 8 }& \multicolumn{2}{c||}{ 9} & \multicolumn{2}{c||}{ 10} & \multicolumn{2}{c||}{ 30}& \multicolumn{2}{c||}{ 30}\\ \hline
& \fe{l�n} & \fe{kat} & \fe{l�n} & \fe{kat} & \fe{l�n} & \fe{kat} & \fe{l�n} & \fe{kat} & \fe{l�n} & \fe{kat} \\ \hline
Video 1 & 20 &c&36 &abh&25 &fij&12& bc&12& b       \\\hline
Video 2 & 5& a&15& f&25& bg&18& i&14& gh           \\\hline
Video 3 & 13& b&4& e&23& fg&13& a&14& ad           \\\hline
Video 4 & 11& b&9 &c&26& bde&9& f&10& i            \\\hline
Video 5 & 10& a&13& c&37& acfij&17& ef&12& h       \\\hline
Video 6 & 10& c&21& dfh&13& ac&16& h&20& cd        \\\hline
Video 7 & 7& i&26& ab&28& afgj&29 &agj&12& gj      \\\hline
Video 8 & 12& b&13& b&29& fij&23& adh&12& df       \\\hline
Video 9 & &&3& e&17& aij&41& agij&17& ae           \\\hline
Video 10 &&&&&32& bdf&31 &eij&20& cgj               \\\hline
Video 11 &&&&&&&17 &cf&19& cgi                      \\\hline
Video 12 &&&&&&&15 &ad&22& bfi                      \\\hline
Video 13 &&&&&&&18 &fg&16& gi                       \\\hline
Video 14 &&&&&&&28 &chi&21& cj                      \\\hline
Video 15 &&&&&&&13 &g&12& j                         \\\hline
Video 16 &&&&&&&7  &b&23& bgh                       \\\hline
Video 17 &&&&&&&10 &e&32& cij                       \\\hline
Video 18 &&&&&&&30 &abh&10& e                       \\\hline
Video 19 &&&&&&&26 &ej&16& dg                       \\\hline
Video 20 &&&&&&&10 &c&21& bgi                       \\\hline
Video 21 &&&&&&&20 &dgh&21& dfg                     \\\hline
Video 22 &&&&&&&22 &fh&13& afg                      \\\hline
Video 23 &&&&&&&7  &dg&24& dh                       \\\hline
Video 24 &&&&&&&19 &af&24& bei                      \\\hline
Video 25 &&&&&&&18 &fj&14& fi                       \\\hline
Video 26 &&&&&&&2  &d&24& aj                        \\\hline
Video 27 &&&&&&&17 &j&34& afij                      \\\hline
Video 28 &&&&&&&24 &ab&15& di                       \\\hline
Video 29 &&&&&&&21 &efg&25& hi                      \\\hline
Video 30 &&&&&&&17 &gi&21& cfi                      \\\hline\hline
\fe{Utdata:} & \multicolumn{2}{c||}{ 33 }& \multicolumn{2}{c||}{ 59} & \multicolumn{2}{c||}{ 79} & \multicolumn{2}{c||}{ 80}& \multicolumn{2}{c||}{ 78} \\ \hline\hline
\end{tabular}
}


\end{document}
