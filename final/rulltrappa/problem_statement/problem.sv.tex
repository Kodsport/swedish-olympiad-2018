\problemname{Rulltrappa}
Paulina Osqulda\footnote{Osqulda är en studentikos benämning på teknologer som studerar vid Kungliga Tekniska högskolan i Stockholm, se \url{https://sv.wikipedia.org/wiki/Osquar_och_Osqulda}.} pluggar på KTH i Stockholm, och åker varje morgon till skolan via tunnelbanan.
När hon kommer fram till sin tunnelbanestation måste hon åka upp längs en rulltrappa för att komma ut.

I rulltrappan bildas det ofta två olika köer.
På den högra sidan av trappan ställer sig folk som vill stå still i rulltrappan, medan man ställer sig på den vänstra sidan om man föredrar att gå i rulltrappan (för att komma upp snabbare).

Paulina har märkt att det oftast bildas en \emph{jättelång} kö till den vänstra sidan av rulltrappan, eftersom alla är så stressade till jobbet och vill kunna gå snabbt upp för rulltrappan.
På sista tiden har hon funderat på om det kanske rentav skulle gå snabbare att istället välja den högra kön, eftersom kön där ofta är kortare.

Rulltrappan är totalt $M$ trappsteg lång.
Om man står i rulltrappan färdas man $S$ trappsteg per sekund uppåt i rulltrappan.
Om man istället går i rulltrappan färdas man $G$ trappsteg per sekund uppåt.

Totalt kan $A$ personer per sekund börja gå i rulltrappan ur den vänstra kön, medan $B$ personer per sekund kan ställa sig på rulltrappan ur den högra kön.
Detta betyder att i början av förloppet går en person på rulltrappan.
Innan en ny person går på rulltrappan måste denna vänta $\frac{1}{A}$ (resp $\frac{1}{B}$ sekunder) för att kunna gå på rulltrappan.

Den vänstra kön är för närvarande $L$ personer lång, och den högra är $R$ personer lång.

Hjälp Paulina avgöra vilken kö hon ska ställa sig i, för att så snabbt som möjligt nå rulltrappans topp.

\section*{Indata}
Den första raden innehåller tre heltal $M, S, G$ ($30 \le M \le 200$, $1 \le S \le G \le M$), antalet steg i rulltrapponerna och antalet steg per sekund man färdas om man står still, respektive går i rultrappan.

Nästa rad innehåller två decimaltal $A, B$ ($0.1 \le A, B \le 1$), antalet personer per sekund som går på den vänstra respektive högra rulltrappan.

Den sista raden innehåller två heltal $L, R$ ($0 \le L, R \le 100$), antalet personer i den vänstra respektiva högre rulltrappan just nu.

\section*{Utdata}
Skriv ut \texttt{latmask} om det snabbaste alternativet är att välja den högra kön och stå i rulltrappan, eller \texttt{friskus} om det snabbaste alternativet är att välja den vänstra kön och gå i rulltrappan.

Det är garanterat att tiden det tar för Paulina att åka upp för rulltrappan skiljer sig med minst 1 sekund mellan de två köerna, så du behöver aldrig bry dig om fallet när båda alternativen är lika snabba.

\section*{Poängsättning}
Din lösning kommer att testas på en mängd testfallsgrupper.
För att få poäng för en grupp så måste du klara alla testfall i gruppen.

\noindent
\begin{tabular}{| l | l | p{12cm} |}
  \hline
  Grupp & Poängvärde & Gränser \\ \hline
    $1$   & $40$     & $L = R = 0$ \\ \hline
    $2$   & $40$     & $S = G$ \\ \hline
    $3$   & $20$     & Inga ytterligare begränsningar. \\ \hline
\end{tabular}

% För att få poäng kommer ditt program behöva klara av testfall både där svaret är \texttt{friskus} och där svaret är \texttt{latmask}.

\section*{Förklaring av exempel 1}
I exempel 1 är båda köer tomma, så Paulina kan omedelbart börja gå upp längs rulltrappan.
Om hon står stilla i rulltrappan färdas hon med 1 trappsteg per sekund, så det tar $\frac{50}{1} = 50$ sekunder.
Om hon istället går upp i rulltrappan färdas hon med 2 trappsteg per sekund, så det tar $\frac{50}{2} = 25$ sekunder, vilket är det snabbare alternativet.

\section*{Förklaring av exempel 2}
I exempel 2 går det inte snabbare för Paulina att gå upp för rulltrappan än om hon står stilla (ty hon har stukat foten).
Däremot har båda sidorna en kö.

I den högra sidan av kön står det 3 personer, och det tar $\frac{3}{0.5} = 6$ sekunder innan dessa personer har gått på rulltrappan.
I den vänstra sidan står det istället 4 personer, och det tar $\frac{7}{1} = 7$ sekunder innan dessa personer har gått på rulltrappan.

Således är det snabbare alternativet att välja den högra kön och stå stilla i rulltrappan.

\section*{Förklaring av exempel 3}
Om Paulina väljer den vänstra kön och går upp för rulltrappan kommer hon upp efter 129 sekunder. 
Om hon istället väljer att ta den högra kön och därmed stå i rulltrappan tar det 130 sekunder.
