\problemname{Trevlig väg}
När jag går hemåt går jag inte alltid den kortaste vägen, utan en väg som
\begin{enumerate}
    \item hela tiden tar mig närmare hem, och
    \item är ''trevligast'', i den meningen att medelvärdet av de passerade vägavsnittens ''trevlighetsfaktor'' är så hög som möjligt.
\end{enumerate}
Skriv ett program som beräknar det maximala sådana medelvärdet.

Kartan över min stad kan beskrivas med hjälp av $n$ platser numrerade från $1$ till $n$. Plats $1$ är min startplats och plats $n$ är mitt hem, och platserna har sorterats efter avstånd så att en plats med högre nummer alltid ligger närmare hemmet än en med lägre nummer.

Vidare finns $m$ olika vägavsnitt som var och en leder från en plats $u_i$ till en annan plats $v_i$ och har en trevlighetsfaktor $w_i$, som kan bero på att där finns några ovanliga träd, någon gullig katt i ett fönster eller något annat trevligt. Eftersom jag alltid vill gå i riktning hemåt, så har vi i beskrivningen endast tagit med vägavsnitt där $u_i<v_i$.

Den som är lite matematiskt intresserad skulle kunna kalla detta för en riktad och viktad acyklisk graf.

\begin{figure}[h]
    \includegraphics[width=7cm]{trevlig.pdf}
    \caption{Kartan i det andra exemplet. Den trevligaste vägen är $1\rightarrow 3\rightarrow 5$.}
\end{figure}

\section*{Indata}
Den första raden innehåller de två heltalen $n$ och $m$ ($2 \leq n \leq 10^5$ , $1 \leq m \leq 2\cdot 10^5$), beskrivna i problemet..

Var och en av de följande $m$ raderna beskriver ett vägavsnitt och innehåller tre heltal $u_i$, $v_i$ och $w_i$ ($1 \leq u_i < v_i \leq n$, $1 \le w_i \le 2\cdot 10^6$),
vilket betyder att vägavsnittet går från plats $u_i$ till plats $v_i$ och har trevlighetsfaktor $w_i$.

Det kommer aldrig finnas mer än ett vägavsnitt som förbinder samma platser, och det garanteras
att det går att ta sig från plats $1$ till plats $n$.

\section*{Utdata}
Skriv ut ett tal: det högsta uppnåbara medelvärdet av trevlighetsfaktorer på en väg från plats $1$ till plats $n$.
Svaret anses korrekt om dess relativa eller absoluta fel är högst $10^{-6}$.

\section*{Poängsättning}
Din lösning kommer att testas på en mängd testfallsgrupper. För att få poäng för en grupp så måste du klara alla testfall i gruppen.

\noindent
\begin{tabular}{| l | l | l |}
\hline
Grupp & Poängvärde & Gränser \\ \hline
$1$     & $21$         &  $2 \le n \le 10$, $1 \le m \le 20$ \\ \hline
$2$     & $41$         &  $2 \le n \le 1000$, $1 \le m \le 2000$ \\ \hline
$3$     & $38$         & Inga ytterligare begränsningar. \\ \hline
\end{tabular}
