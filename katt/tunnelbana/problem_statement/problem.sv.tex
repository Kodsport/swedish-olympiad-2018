\problemname{Tunnelbana}
I Stomholck är tunnelbanenätet format som ett träd, och till skillnad från bussarna kommer tunnelbanan oftast i tid.
Du planerar att genomföra $m$ st resor i tunnelbanenätet, och vill göra det så billigt som möjligt.
 
Kostnaden för att resa mellan två stationer är 1 krona per kant på vägen mellan stationerna. 
Det går dessutom att köpa ett kort som tillåter obegränsat antal resor på alla kanter mellan två valfria stationer utan extra kostnad.
Kortets kostnad är $k$ kronor per kant på den valda vägen och en kund får inte köpa mer än ett kort.
Man behöver inte köpa ett kort om man inte vill. Eftersom nätet är ett träd finns det alltid exakt en väg mellan varje par av noder.

Given ett nätverk med $n$ stationer och $m$ resor, avgör den minsta kostnaden att utföra resorna.

\section*{Indata}
Den första raden innehåller tre heltal $n$, $m$ och $k$ ($2 \leq n \leq 10^5$, $0 \leq m, k \leq 10^5$).
De följande $n-1$ raderna innehåller två heltal $u_i$ och $v_i$ ($1 \leq u_i , v_i \leq n$ , $u_i \neq v_i$), vilket betyder
att en kant går mellan noderna $u_i$ och $v_i$.
De följande $m$ raderna innehåller två heltal $a_i$ och $b_i$ ($1 \leq a_i , b_i \leq n$ , $a_i \neq b_i$), vilket betyder 
att resa nummer $i$ går mellan $a_i$ och $b_i$.

\section*{Utdata}
Skriv ut ett heltal, den minsta kostnaden för en person att resa alla $m$ resor.

\section*{Poängsättning}
Din lösning kommer att testas på en mängd testfallsgrupper. För att få poäng för en grupp så måste du klara alla testfall i gruppen.

\noindent
\begin{tabular}{| l | l | p{12cm} |}
  \hline
  \textbf{Grupp} & \textbf{Poäng} & \textbf{Gränser} \\ \hline
  $1$    & $19$      & $2 \le n, m \le 50$ \\ \hline
  $2$    & $26$      & $2 \le n, m \le 1000$ \\ \hline
  $3$    & $20$      & $2 \le n, \le 20000$ , $0 \le m \le 1000$ \\ \hline
  $4$    & $35$      & Inga ytterligare begräsningar. \\ \hline
\end{tabular}

\section*{Förklaring av Sample 1}
Om man inte köper något kort blir kostnaden för de två resorna $7$. Om man däremot köper
ett kort mellan $2$ och $5$ tjänar man två kronor, så kostnaden blir $5$.

\section*{Förklaring av Sample 2}
Här är det inte värt att köpa något kort alls.
