\problemname{Lisp till C}
C är ett gammalt programmeringsspråk från 70-talet, som trots dess ålder är väldigt välanvänt.
I språket skrivs funktionsanrop till en funktion med namnet \texttt{namn} på formen \texttt{namn(parameter1, parameter2, ..., parameterN)}.
En parameter kan antingen vara en variabel (som i vårt fall består av en sekvens av tecken \texttt{a-z}), eller ett funktionsanrop i sig.
T.ex. kan ett fullständigt funktionsanrop se ut på följande vis:
\begin{verbatim}
a(b, c(d), e(f, g(h, i, j)))
\end{verbatim}
Ett funktionsanrop utan parametrar skrivs som \texttt{funktion()}.

Lisp är ett annat programmeringsspråk från 50-talet.
I språket skrivs funktionsanrop till en funktion med namnet \texttt{namn} på formen \texttt{(namn parameter1 parameter2 ... parameter N)}.
Återigen kan en parameter i sig vara en variabel eller ett funktionsanrop.
Funktionsanropet i C som beskrivs ovan kan istället skrivas
\begin{verbatim}
(a b (c d) (e f (g h i j)))
\end{verbatim}
Ett funktionsanrop utan parametrar skrivs som \texttt{(funktion)}.

Varför denna språkhistoria?
Jo, det visar sig att domaren Simon gillar C, men inte Lisp.
Domaren Johan, å andra sidan, gillar Lisp, men inte C.

Simon blev därför väldigt sur när Johan programmerade alla sina exempellösningar i Lisp, och vill konvertera Johans program till C.
Hjälp Simon med detta, genom att skriva ett program som, givet ett funktionsanrop i Lisp konverterar det till ett funktionsanrop i C.

\section*{Indata}
Indatan består av en rad med ett korrekt formaterat funktionsanrop i Lisp-format, högst $100\,000$ tecken lång.
Alla parametrar kommer att vara separerade med exakt ett blanksteg, och det finns inga extra blanksteg där det inte behövs.

\section*{Utdata}
Skriv ut en enda rad, med funktionsanropet konverterat till C-format.
Inkludera ett blanksteg efter varje kommatecken, som i exempelfallen.

\section*{Poängsättning}
Din lösning kommer att testas på en mängd testfallsgrupper. För att få poäng för en grupp så måste du klara alla testfall i gruppen.

\noindent
\begin{tabular}{| l | l | p{12cm} |}
  \hline
  \textbf{Grupp} & \textbf{Poäng} & \textbf{Gränser} \\ \hline
  $1$    & $16$      & Ingen parameter i indatan är ett funktionsanrop \\ \hline
  $2$    & $27$      & Anropen är nästlade högst två nivåer: \texttt{(a (b) (c))} och \texttt{(a (b c))} kan förekomma, men inte \texttt{(a (b (c)))} \\ \hline
  $3$    & $15$      & Alla anrop har minst en parameter, och alla variabler och funktionsnamn består av enbart en bokstav \\ \hline
  $4$    & $28$      & Indatan är högst $100$ tecken lång \\ \hline
  $5$    & $14$      & Inga ytterligare begränsningar  \\ \hline
\end{tabular}


